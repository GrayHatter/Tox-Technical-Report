\documentclass{tox}

\begin{document}

\title{A secure, distributed multimedia messenger}

\abstract{\
  \textbf{Abstract}:
  This is a brief paragraph about Tox, how it works, what technologies it uses and about this technical paper in general.
  Lorem ipsum dolor sit amet, consectetur adipisicing elit, sed do eiusmod tempor incididunt ut labore et dolore magna aliqua.
  Ut enim ad minim veniam, quis nostrud exercitation ullamco laboris nisi ut aliquip ex ea commodo consequat.
  Duis aute irure dolor in reprehenderit in voluptate velit esse cillum dolore eu fugiat nulla pariatur.
  Excepteur sint occaecat cupidatat non proident, sunt in culpa qui officia deserunt mollit anim id est laborum.
}

\author{Written by the Tox Technical Reporting working group}

\maketitle

\section{TCP Network}

It has come to our attention that to achieve decent market penetration Tox must work behind ALL internet connections, may they be behind enterprise NATs or any other bad network conditions. 

The people who have issues with the UDP direct connection approach seem to be a small minority though it is hard to estimate how many.

This means that routing their packets using good nodes on the network will probably not take a huge toll on the network and will assure that peoplecan use Tox regardless of the quality of their internet connection.

How it's going to work:

1. Alice, a Tox client on a TCP only network generates a temporary public key and connects to a bootstrap node.
2. Using the bootstrap node she finds and connects to a couple (exact number to be determined later) number of random nodes that have TCP relay support.
3. She uses the onion through the TCP relay connections to send friend requests or tell online friends which TCP nodes she is connected to and her temporary public key.
4. Bob receives an onion packet from Alice telling him which nodes she is connected to. Bob connects to these nodes and establishes a routed connection with alice using that temporary public key.
5. That connection is used by both to transmit encrypted Messenger and A/V packets.
6. If one of the nodes shuts down while it is currently routing traffic, Alice and bob just switch to one of the other nodes they are both connected to.

Detailed implementation details:

There are two distinct parts for TCP relays, the client part and the server part.
The server acts as the actual relay. Servers must have fully forwarded TCP ports (NAT-PMP and uPNP can help here). The first port the server will try binding to is 443 followed by port 3389 and possibly some others. Onion packets can be sent/received through the TCP servers.

Server:

The public/private key pair the TCP server uses is the same one he uses for the DHT. 

all crypto for communication with the server uses the crypto_box() function of NaCl.

TCP doesn't have packets so what we will refer to as packets are sent this way: [[uint16_t (length of data)][data]]

So if you would inspect the TCP stream you would see:[[uint16_t (length of data)][data]][[uint16_t (length of data)][data]][[uint16_t (length of data)][data]]

Note that both handshake packets don't have this format (the length for them is always the same so we don't need to specify it.)

When the client connects to the server, he sends this packet:[public key of client (32 bytes)][nonce for the encrypted data [24 bytes]][encrypted with the private key of the client and public key of the server and the nonce:[public key (32 bytes) and][base nonce we want the server to use to encrypt the packets sent to us (24 bytes)]]

The server responds with:[nonce for the encrypted data [24 bytes]][encrypted with the public key of the client and private key of the server and the nonce:[public key (32 bytes) and][base nonce we want the client to use to encrypt the packets sent to us (24 bytes)]]

All packets to the server are end to end encrypted with the information received (and sent) in the handshake.

(first packet is encrypted with the base nonce the private key for which the client sent the server the public key and the public key we sent to the client, the next with base nonce + 1...)

The connection is set to an unconfirmed state until a packet is received and decrypted correctly using the information in the handshake.
each packet sent to/from the server has an id (the first byte of the plain text data of the packet.)
ids 0 to 15 are reserved for special packets, ids 16 to 255 are used to denote who we want the data to be routed to/who the packet is from.

special ids and packets:0 - Routing request.[uint8_t id (0)][public key (32 bytes)]

1 - Routing request response.[uint8_t id (1)][uint8_t (rpid) 0 if refused, packet id if accepted][public key (32 bytes)]
2 - Connect notification:[uint8_t id (2)][uint8_t (packet id of connection that got connected)]
3 - Disconnect notification:[uint8_t id (3)][uint8_t (packet id of connection that got disconnected)]
4 - ping packet[uint8_t id (4)][uint64_t ping_id (0 is invalid)]
5 - ping response (pong)[uint8_t id (5)][uint64_t ping_id (0 is invalid)]
6 - OOB send[uint8_t id (6)][destination public key (32 bytes)][data]
7 - OOB recv[uint8_t id (7)][senders public key (32 bytes)][data]
8 - onion packet (same format as initial onion packet (See: Prevent tracking.txt) but packet id is 8 instead of 128)
9 - onion packet response (same format as onion packet with id 142 but id is 9 instead.)

The rest of the special ids are reserved for possible future usage.

If the server receives a routing request he stores server side that the client wants to connect to the person with that public key and sends back a Routing request response with the rpid along with the public key sent in the request.
If for some reason the server must refuse the routing request (too many) he sends the response with a rpid of 0.
If the person who the client wants to connect to is also online and wants to connect to the client a connect notification is sent to both with the appropriate packet id.
If either one disconnects, a disconnect notification is sent to the other with appropriate packet id.
If a client sends a disconnect notification, the entry on the server for that routed connection is cleared and a disconnect notification is sent to the peer (if he was online)
If the server receives an onion packet he handles it the same as he would if it was one received normally via UDP, he must also assure himself that any responses must be sent to the proper client.
Ping responses must have the same ping_id as the request.
If the server receives a ping packet he must respond with a ping response.
The server will send a ping packet to clients every 30 seconds, they have 30 seconds to respond, if they don't the connection is deleted.
OOB send packets will be sent to the peer connected to the TCP server with the destination public key as a OOB recv packet. The client sending this packet has no way of knowing if the packet reached its destination.


\end{document}
