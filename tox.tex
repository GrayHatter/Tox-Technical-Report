\documentclass{tox}

\begin{document}

\title{A secure, distributed multimedia messenger}

\abstract{\
  \textbf{Abstract}:
Tox is a secure, lightweight, distributed messenger platform designed to 
provide maximum security without penalizing performance.
}

\author{Written by irungentoo} %add your name here when you contribute to it.

\maketitle

\tableofcontents
\clearpage

\section{Introduction}

When one first starts a Tox client, a long term asymmetric keypair is created. 
This keypair identifies the user uniquely. If a Tox user wants to connect to 
another user, he needs to obtain their long term public key. 

\subsection{Crypto}

Tox uses crypto\_box() from the NaCl crypto library for all the cryptography in 
Tox. Unless otherwise noted, all keys refer to keys generated with 
crypto\_box\_keypair(), all encryption is done with crypto\_box() and all 
decryption with crypto\_box\_open(). For performance purposes the functions to 
precomputed the shared secret and encrypt and decrypt messages with it are used 
extensively in Tox however this is not relevant to this document.

The function crypto\_box() provides fast public-key authenticated encryption. 
For exactly how it works read the NaCl docs.

\subsection{Tox ID}

The Tox ID is the long term public key of that Tox user with a nospam value and 
a checksum appended to it. The exact format is:

[NaCl crypto\_box() public key (32 bytes)][nospam (4 bytes)][checksum (2 bytes)]

The Tox ID is 38 bytes and usually distributed in hex format. A typical valid 
Tox ID looks like:

56A1ADE4B65B86BCD51CC73E2CD4E542179F47959FE3E0E21B4B0ACDADE51855D34D34D37CB5

The nospam value is there to permit users from quickly changing their Tox IDs 
without changing identities. How it works is explained later in the friend 
request section.

The checksum prevents users from adding mistyped or badly copied Tox IDs. It is 
a simple XOR checksum. Code to compute the checksum from a partial ID of 
ID\_length follows.

\begin{lstlisting}
uint8_t checksum[2] = {0};
uint32_t i;

for (i = 0; i < ID_length; ++i)
    checksum[i % 2] ^= ID[i];

\end{lstlisting}

\subsection{Friends}

To add another Tox user as a friend, the user must first obtain their Tox ID. 
The Tox ID can be transmitted across insecure channels or posted publicly. The 
user must make sure however that the Tox ID he adds belongs to his friend and 
not an impersonator.

Each Tox instance contains a friends list. This is a list of added friends 
which the user has added to his client. Tox will try to connect to everyone in 
the friends list. In order to connect to a friend, the friend must have the 
user in their friends list or else the connection will fail. Friends who the 
user is connected to are shown as online, friends who are not connected are 
shown as offline.

Users who are not in the friends list of a certain user cannot connect to him.

There are two ways to add an entry to the friends list. The first is to add the 
Tox ID of a Tox user and the second is to accept a friend request.

\subsection{Friend Requests}

When a Tox user adds a Tox ID, Tox sends a friend request to the Tox user 
corresponding to the added ID. A friend request contains the long term public 
key of the user making the request, the nospam contained in the Tox ID and a 
custom user specified message.

When a Tox user receives a friend request, he checks if the nospam is valid (if 
it matches the one he used to generate a Tox id). If it isn't valid, the friend 
request is discarded. If it is valid, the friend request is presented to the 
user along with the message specified in it.

The user can then accept the request and add the other user to his friends list 
or he can refuse it and simply discard the friend request.

\subsection{Text Communications}

Tox users have various ways of communication to online friends. The first is 
basic text communication. Tox uses UTF-8 as the encoding for all text 
transmitted through it. UTF-8 means extensive language support and support for 
unicode smileys and other characters.

Each Tox user has a nickname and a status that can be changed at will. They are 
transmitted to friends every time a connection to them is established and to 
all currently online friends when the user changes it.

Tox messages can be sent as normal messages or as actions (the equivalent of 
IRC's /me).

\subsection{File Transfers}

Tox supports sending files to online friends. Users can pause and resume files 
when they are transmitted. If clients become disconnected while the transfer is 
in progress then reconnect after, Tox supports resuming the file transfer.

File transfers in Tox are much faster than other typical messaging software due 
to the fact that it connects directly to other peers most of the time.

\subsection{Audio/Video}

Tox users can make audio and video calls to other Tox users. Call settings can 
be modified during the call which means a user can stop sending video or audio 
while a call is in progress. The video resolution can be changed dynamically 
during a video call.

\subsection{Conclusion}

This introduction was meant to give the reader a quick high level overview of 
Tox: the part that directly interacts with the user. The next section will deal 
with how Tox works at a lower level.

\section{Deeper Into Tox}

The Tox core can be broken down into 2 parts: The peer finding part and the 
messenger part. The peer finding part takes care of finding the peer (friend) 
we 
want to establish a secure connection to and the messenger part takes care of 
connecting to and communicating with the peer after the peer finding succeeds.

\subsection{Peer Finding}

The peer finding part has the purpose of finding peers inside the network. It 
is the part that finds the friends in the friend list of the Tox client and 
sends them the required information so that both can establish a secure 
connection to each other.

Peer finding has two parts: the onion and the DHT which work closely together.

The Tox DHT is a UDP only DHT that Tox clients advertise themselves to with a 
temporary public key called the DHT public key. The DHT public key is 
regenerated every time the Tox client is restarted to prevent the tracking of 
Tox clients inside the network.

Because the DHT only works with temporary keys, there needed to be a way to 
communicate to peers when all that was known was their real long term public 
key. This part is named the onion. The onion is used to anonymously send and 
receive friend requests and packets from friends containing what their current 
DHT public key is along with some TCP relays they are currently connected to.

Once the peer knows the DHT public key of a friend he queries the DHT then 
finds and connects to the friend. In case the friend cannot be found in the DHT 
Tox falls back to using TCP relays to connect to the friend by connecting to 
the relays the friend is connected to and using them to relay data to the 
friend.

The peer finding succeeds when the Tox client has found the best way to send 
data to the friend.

\end{document}
