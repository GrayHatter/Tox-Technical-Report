\documentclass{tox}

\begin{document}

\title{A secure, distributed multimedia messenger}

\abstract{\
  \textbf{Abstract}:
  This is a brief paragraph about Tox, how it works, what technologies it uses and about this technical paper in general.
  Lorem ipsum dolor sit amet, consectetur adipisicing elit, sed do eiusmod tempor incididunt ut labore et dolore magna aliqua.
  Ut enim ad minim veniam, quis nostrud exercitation ullamco laboris nisi ut aliquip ex ea commodo consequat.
  Duis aute irure dolor in reprehenderit in voluptate velit esse cillum dolore eu fugiat nulla pariatur.
  Excepteur sint occaecat cupidatat non proident, sunt in culpa qui officia deserunt mollit anim id est laborum.
}

\author{Written by the Tox development team}

\maketitle

% /g/ asked for math, leave it be here until we get some
\newtheorem{thm}{Theorem}
\begin{thm}
  Let $a\in\mathbb{Z}$ and $d\in\mathbb{N}\setminus\{0\}$.
  Then there is exactly one $(q,r)\in\mathbb{Z}\times\mathbb{N}$ such that $a = qd+r$ and $r<d$.
\end{thm}

\begin{proof}
  Let $q\in\mathbb{Z}$ maximal such that $q\leq a/d$ and $r=a-qd$.
  Then $(q,r)$ satisfies the equation.
  Furthermore $r<d$ since otherwise
  \begin{align*}
    (q+1)d \leq qd+r = a
  \end{align*}
  which contradicts the maximality of $q$.

  Assume that $(\tilde q,\tilde r)$ satisfies the same conditions.
  Without loss of generality we can assume $\tilde r\leq r$.
  Then
  \begin{align*}
    0\leq |\tilde q-q|d = r-\tilde r \leq r < d.
  \end{align*}
  Hence $|\tilde q-q|\in\mathbb{N}$ is strictly smaller than $1$ and thus $\tilde q = q$.
  It is clear that $\tilde r=r$.
\end{proof}

\end{document}
