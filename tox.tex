\documentclass{tox}

\begin{document}

\title{A secure, distributed multimedia messenger}

\abstract{\
  \textbf{Abstract}:
  This is a brief paragraph about Tox, how it works, what technologies it uses and about this technical paper in general.
  Lorem ipsum dolor sit amet, consectetur adipisicing elit, sed do eiusmod tempor incididunt ut labore et dolore magna aliqua.
  Ut enim ad minim veniam, quis nostrud exercitation ullamco laboris nisi ut aliquip ex ea commodo consequat.
  Duis aute irure dolor in reprehenderit in voluptate velit esse cillum dolore eu fugiat nulla pariatur.
  Excepteur sint occaecat cupidatat non proident, sunt in culpa qui officia deserunt mollit anim id est laborum.
}

\author{Written by the Tox development team}

\maketitle

% /g/ asked for math, leave it be here until we get some
Division-with-remainder theorem (enjoy /g/)
\newline
\newline
Given $$a \in \mathbb{Z}$$and $$d \in \mathbb{N}$$then there exist unique $$q,r \in \mathbb{Z}$$ such that $$a = qd + r$$ and $$0 <= r < d$$
\newline
Proof
\newline
Let q be the largest integer such that $$q <= a/d$$ and let $$r = a - qd$$ $$a = qd +r$$ $$qd <= a$$ $$0 <= a - qd = r$$ Since q uses the largest integer, $$q + 1 > a/d$$ $$qd + d > a$$ $$=> r = a -qd < d$$ Proving uniqueness, suppose $$a = q_1d + r_1$$ $$a = q_2d + r_2$$ for $$q_1,q_2 \in \mathbb{Z}$$ and $$r_1,r_2 \in {0,1,...,d-1}$$ and $$r_1 >= r_2$$ then $$0 <= r_1 -r_2 <= r_1 <= d -1$$ $$r_1 -r_2 = (a -q_1d) -(a -q_2d)$$ $$r_1 -r_2 = a -q_1d -a + q_2d = d(q_1 -q_2)$$ However, only multiple of d in $$0,1,..,d-1$$ is $$0$$ therefore $$r_1 -r_2 = 0$$ therefore $$q_2 -q_1 = 0$$

Page 1

\end{document}
